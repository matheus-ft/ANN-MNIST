\section{Implementação da Rede neural}
Para implementação da rede neural utilizamos como fonte de consulta, principalmente, os slides e as notas de aula. Ao inicio da implementação tínhamos uma rede neural construída pelo professor e apresentada em aula para utilizar como esqueleto. Todavia esta rede neural não funcionava para um numero qualquer de camadas escondidas e de neurônios. A ideia aplicada foi salvar cada um dos elementos da rede neural, em uma lista que poderia conter uma quantidade qualquer daquele elemento. Por exemplo, para as ativações, criamos uma lista $a$, na qual cada um de seus elementos é uma matriz de ativação cujo ínidce na lista corresponde à camada a que pertence. Ainda assim,  guardar os dados em listas não foi suficiente, pois como as funções  foram construídas para uma única camada escondida, algumas das funções tiveram que ser alteradas e outras criadas, a fim de realizar novos procedimentos além daqueles prontos. Como o cálculo do erro nos neurônios não é realizado para o \textit{bias}, a função \textit{gradientDescent} foi alterada para que não considerasse o bias da camada $i$, ao calcular o bias da camada $i-1$, desta forma a função passou a função a funcionar para um número qualquer de camadas escondidas, as quais, por sua vez, podem possuir um número qualquer de neurônios.