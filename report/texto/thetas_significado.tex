\section{Representação Visual dos Thetas}
Nota-se que como demonstrado no artigo \cite{voss2021visualizing} existem diversas formas de vizualizar os pesos para problemas que envolvem previsões utilizando imagens. A função que desenvolvemos é pesadamente inspirada num \textit{snippet} encontrado ao pesquisar modos de fazer a visualização dos \textit{tethas} mais interessante \cite{Snaped}, e, como resultado, obtivemos uma imagem para unidade de ativação na primeira camada escondida, formadas ao reconstruir a matriz $20\times20$ a partir de cada linha da matriz de pesos $\Theta^{(1)}$. Esta imagem é uma representação de um "filtro" pois diz respeito a qual forma ou região da figura esta sendo analisada. Ou seja, conclui-se que os pesos dizem respeito ao que aquele neurônio está extraindo de uma imagem dada, e, por consequência, a ativação do neurônio a qual estas \textit{features} estão conectadas será um valor que representa o quanto da forma está contida na imagem original.